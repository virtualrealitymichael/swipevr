\section{Related Work}

In this section, we discuss studies applicable to virtual reality based text entry.
For a full review of text entry methods, see~\cite{mackenzie2002text, zhai2005search, Buxton2011}.

\subsection{Novel QWERTY Keyboards}
Researchers leverage the familiarity of the QUERTY~\cite{noyes1983qwerty} keyboard with new text entry methods.
The Half-QWERTY keyboard~\cite{matias1993half} is arm-mounted and has a reduced size by using only half of a QWERTY keyboard.
The Pinch Gloves~\cite{bowman2002text} controls a virtual QUERTY keyboard with hand rotations and finger-mounted buttons.
Another glove, KITTY~\cite{Kuester:2005:TKI:1101616.1101635} uses the tapping of contacts on each finger with the thumb to mimic a keyboard.
DigiTap~\cite{Pratorius:2014:DEV:2671015.2671029} is a similar idea but uses a wrist mounted camera instead of a glove.

\subsection{Gloves, Chorded  Keyboards, and Touch Pads}
Other approaches abandon the QWERTY keyboard entirety.
Chorded keyboards~\cite{noyes1983chord}, such as Twiddler~\cite{lyons2004twiddler} use a combination of a few keys to produce characters.
Chording is done with gloves, such as Chording Glove~\cite{rosenberg1999chording}.
The VirtualPhonepad~\cite{ahn2006virtualphonepad} imitates text entry via a $3\times3$ mobile phone keypad.
Virtual Notepad~\cite{poupyrev1998virtual} allows for creating handwritten annotations in virtual reality.
Connecting the Dots~\cite{frees2006connecting} approximates handwriting by having users connect dots on a $3\times3$ virtual dot matrix to scribe a single character.

In a literature review, the Pinch Gloves, pen and tablet, chorded keyboard, and a speech are compared~\cite{bowman2002text}.
The pen and tablet and speech are fastest.  The pen and tablet had the fewest errors.
The comparison was done before specialized virtual reality input device were invented.

\subsection{Word Gesture Keyboards}
Word gesture keyboards such as ShapeWriter~\cite{zhai2012word,Zhai:2009:SIL:1520340.1520380} and SHARK 2~\cite{kristensson2004shark}, are gesture based text input methods.
Words are written through a gesture that links the letters of the intended word on a soft keyboard.
A language model selects the most likely word from a list of possible candidates.
For out of vocabulary and hard-to-predict words, a deterministic input method is still needed.
VelociTap~\cite{vertanen2015velocitap} takes the predictive model further by only predicting the whole sentence.

\subsection{Home Entertainment Input Devices}
Text entry methods is sought after in other application areas, such as home entertainment systems.
SpeeG~\cite{hoste2012speeg} is a Kinect~\cite{geerse2015kinematic} version of Speech Dasher~\cite{vertanen2010speech} text input technique.
The user speaks and then corrects mistakes using an interface that scrolls through the recognized text as well as alternatives.
SpeeG2~\cite{hoste2013speeg2} uses a grid instead of a scroll.

\subsection{Speech Recognition}
Spoken language is effective for human-human interaction but often has limitations when applied to human-computer interaction.
Speech is slow for presenting information and interferes significantly with other cognitive tasks~\cite{shneiderman2000limits}.

Speech recognition is limited when typing passwords, names, and URLs~\cite{bazzi2002modelling}. 
A lack of privacy, the perception of bothering others, and the awkwardness of speaking to a machine~\cite{sawhney2000nomadic} restrict the use cases in which speech is useful.
Some users find speech unreliable when an accent or ambient sound negatively affect accuracy.
Studies have shown that users find it ``harder to talk and think than type and think'' and considered the keyboard to be more ``natural'' than speech for text entry ~\cite{Karat:1999:PEC:302979.303160}.

A virtual reality speech-based text-entry system represents candidate words recognized by the speech recognition engine as blocks that can be moved around in the virtual enviroment~\cite{osawa2002multimodal}. 

For mobile phones, speech entry is three times faster than typing for English and Mandarin~\cite{ruan2016speech}.  
Participants were given a set of phrases to type and speak.
However, normal speech is filled with hesitations, repetitions, changes of subject in the middle of an utterance~\cite{forsberg2003speech} as the user thinks.
This could lead to an unrealistic accuracy and speed for speech.  
The study excludes punctuation marks and capital letters from the phrases.
Generating accurate punctuation  automatically with speech input is not yet possible and must be done manually~\cite{chen1999speech}.

\subsection{Gaze Directed Typing}
Gaze directed typing ~\cite{majaranta2009text} places a mouse pointer directly at the user's point of gaze over a virtual keyboard.
This is the main text input method in systems such as Oculus GearVR~\cite{hecht2016optical}.
An advantage of this method is that it only uses the hardware available in the head-mounted display without the need for additional controllers.
This method is slow~\cite{majaranta2002twenty, card1983psychology, mackenzie1992fitts} and inconvenient since the whole head has to move to select each character.
Moving the head in a virtual enviroment while the soft keyboard remains static can causes nausea and dizziness~\cite{atienza2016interaction}.

\subsection{Virtual Reality Controllers}
Input methods and controllers are being developed specifically for use in virtual reality.
FaceTouch~\cite{Gugenheimer:2016:FTI:2851581.2890242} uses the backside of the head mounted display as a touch sensitive surface.
The user selects virtual content by touching the corresponding location on backside of the display.
It is useful for short periods of time until the users' arms get tired.

Belt~\cite{dobbelstein2015belt} is an unobtrusive input device for wearable displays with a touch surface encircling the users' hips.
The device suffers from low throughput, arm fatigue, and social acceptance.
Nenya~\cite{ashbrook2011nenya} uses a finger ring as an input mechanism that is always available.
It is fast to access, allows analog input, and is socially acceptable by being embodied in a commonly worn item.

HTC, Oculus, and Google have each released a virtual reality controller design, shown in Figure~~\ref{fig:controllers}.
They are similar in that they have rotational tracking, a way to track the thumb through either joystick or touchpad, and at least four buttons.
They are different in that HTC's Vive and Oculus' Touch controllers use an absolute tracking system to minimizing sensor drift and to allow for rotational and translational tracking~\cite{hilfert2016low}.
Google's Daydream controller is a unimanual, whereas they other two are bimanual.
Slide, the text entry system described in this paper, is implemented using the Vive.
The other two controllers also have the hardware capabilites to support Slide.

\newcommand{\ra}[1]{\renewcommand{\arraystretch}{#1}}



\begin{comment}
\begin{tabular}{@{}rrrrcrrrcrrr@{}}\toprule
& \multicolumn{3}{c}{$w = 8$} & \phantom{abc}& \multicolumn{3}{c}{$w = 16$} &
\phantom{abc} & \multicolumn{3}{c}{$w = 32$}\\
\cmidrule{2-4} \cmidrule{6-8} \cmidrule{10-12}
\end{comment}


\begin{table*}\centering
\ra{1.3}
\begin{tabular}{@{}rrrrr@{}}\toprule
&&&\multicolumn{2}{c}{Virtual~Reality} \\
\cmidrule{4-5}

&$Speech~\cite{ruan2016speech}$     &$Mobile~\cite{ruan2016speech}$ &$Gaze~\cite{majaranta2006effects}$ &$Slide$    \\
\midrule
$WPM$         & 161     & 53      & 7       & 34        \\
$Error~Rate$      & 2.93\%    & 3.68\%    & .54\%     & 4\%       \\
%$Positioning$      &       & absolute    & absolute    & relative        \\
$Hardware$        &microphone   & touch screen  & HMD \& button & controller    \\
$Keys$          &       & 26+     & 26+     & 6         \\
$Special~Characters$  &       & \checkmark  & \checkmark  & \checkmark    \\
$Granulatiry$       &word     & character   & character   & word or character \\
$Hands$         & 0       & 1 or 2    & 1       & 1         \\
$Real-Time~Feedback$  & word      & character   & character   & word        \\
$Cursor~Mode$       & NA      & NA      & persistent  & snap-to-home    \\
$Silent$        &       & \checkmark  & \checkmark  &\checkmark     \\
$Muscle$        & vocal chords  & fingers   & neck      & thumb     \\


\bottomrule
\end{tabular}
\caption{A table comparing speech, mobile keyboard, gaze, and Slide.
Slide is is the text entry method presented in the paper.
For comparison, gaze is the main text entry method in use in systems such as Oculus GearVR.
}
\label{table:comparison}
\end{table*}

\begin{figure}
  \centering
  \begin{subfigure}{.4\columnwidth}
  \includegraphics[width=\textwidth]{figures/controllerVive}
  \caption{HTC's Vive }\label{fig:controllerVive}
  \end{subfigure}
  \\
  \begin{subfigure}{.4\columnwidth}
  \includegraphics[width=\textwidth]{figures/controllerOculus}
  \caption{Facebook's Oculus}\label{fig:controllerOculus}
  \end{subfigure}
  \\
  \begin{subfigure}{.4\columnwidth}
  \includegraphics[width=\textwidth]{figures/controllerDaydream}
  \caption{Google's Daydream}\label{fig:controllerDaydream}
  \end{subfigure}
  \caption{
  The three major commercially available controllers are shown above.
  All contain a way to track the thumb, the necessary hardware needed to implement the text entry system described in this paper.
  }~\label{fig:controllers}

\end{figure}



