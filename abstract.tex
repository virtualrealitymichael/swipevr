\begin{abstract}
Text input in VR is challenging because of the latency in visual feedback and the lack of proprioception.  Our design Slide, a virtual keyboard that requires only a track pad, is based on 3 concepts: (1) self-relative vectors: to enter a character, the user slides from what is assumed to be the centroid of the keyboard to the key of interest.  (2) directional, with error correction: all keys are grouped in 6 tiles, and the user only has to slide in the direction of the tile containing the key.  A Bayesian word recommender determines the word entered. (3) Intention-based hybrid keyboard: the keyboard automatically switches between the 26-key and 6-tile keyboard based on the user's finger movements.   Our user study suggests that Slide can be learned quickly, and delivers an average text entry speed of Y WPM with an error rate of Z\%.  




\begin{comment}
 is able to be learned quickly by beginners and supports eyes-free entry by experts.  The participants are able to type an average text entry speed of Y WPM with an error rate of Z\%.  These results show that text entry is similar to that of a mobile phone keyboard but error correction and editing, the most time consuming step, is greatly improved.
\end{comment}
\end{abstract}

\category{H.5.1.}
{Information Interfaces and Presentation (e.g. HCI)}
{Multimedia Information Systems}
\category{I.3.7.}
{Computer Graphics}
{Three-Dimensional Graphics and Realism}

%link at: http://www.acm.org/about/class/ccs98-html

\keywords{\plainkeywords}


