\section{Conclusion}

\subsection{Overcomes Latency and Proprioception with Slide}


The lack of proproception in virtual reality is compounded by the latency in visual feedback.
For a user to be able to steer their finger to a key, the time needed to transmit information from the real to virtual and back to the ``real'' through the  screen and into the eyes of the user is too long.  Other commonly used input methods, such as gesture keyboards, are subject to the same problem.
Therefore, we devise a method of input that does not depend on a feedback loop that requires the user to reach an absolute position on a keyboard.  
Our technique only depends on relative movement. The user slides the finger on the touchpad to indicate the relative position of the key of interest with respect to the center of the keyboard.  We further reduce the precision required by grouping keys into 6 tiles, and the user only needs to indicate the direction of the tile, again, with respect to the center of the keyboard.
This reframes the domain of text input from a closed  feedback control problem to an open feedback control problem.
As such we have created a system that lets users enter text in virtual reality efficiently.  With the help of error correction, and including the time of error correction, users can type at about 34 words per minute with an error rate of 4\%.  Furthermore, when precision is desired, the user can simply go slower and hit the precise keys.  The user can still precisely, which is useful for passwords, etc, at a rate of 12 words per minute. 

\subsection{Smooth Transition From Newbie to Novice to Expert }

A smooth pathway learnability is an important factor for an input device to gain widespread adoptability.
Some users will immediately reject an input device that requires too much up front practice before it becomes useful.
Slide has a natural path to transition novice (low threshold) to expert (high ceiling) users~\cite{grover2013computational}.

Slide offers a clear path from newbie to novice to expert. 
Slide is instantly usable: newbie users can use the system quickly by looking at the controller and slowly moving their thumb to a key.
Novice users would periodically glance down to look at the virtual keyboard if they forget the position of a key. 
With a little of practice, users can remember the location of the keys.  
Experts can just type without having to look at the virtual keyboard. 

\subsection{General-Purpose Text Entry}
As virtual reality matures, we expect VR systems to have a standard text-entry method at the OS level, that can be learned once and used across applications. 
This is analogous to the adoption of a pointing device for PCs.  This text-entry method must be general-purpose, accommodating the entry of 
proper nouns, abbreviations, capital letters, colloquialism, passwords, etc. 
Allowing both fast entry of common words and precise, albeit slower, of the characters, Slide is positioned well as a candidate for the keyboard of a VR operating system.        

\begin{comment}
\subsection{Accessibility}
Virtual reality gives access access to something that they might never see in real life.
But for a disabled person, virtual reality might be the path to inclusion.
SwipeVR can be adapted for assistive technology products such as wands, joysticks, trackballs, touch screens.


\subsection{End}

concluding notes
\\
We present a fast text entry input device for virtual reality.
It is faster than the state of the art.
It can be used in public.
It dosnt tire your arms or vocal chords.

Using virtual reality for programming is a rich area for further discovery.
The large workspace that virtual reality provides is received well by users.

They keyboard is currently the weak point in many virtual reality applications.

Touch-typing on the keyboard is difficult as is switching context between the virtual reality controllers and the keyboard.

In the future, high-quality virtual reality headsets will be standalone devices, not tethered to desktop computers \cite{schaller1997moore}.
As the price of headsets decrease \cite{brown2016virtual}, virtual reality is positioned to become a democratizing technology.
Thus it is a worthwhile endeavour to determine how to program in virtual reality, both as a teaching tool for novices and as a more expressive tool for experts.

\end{comment}
