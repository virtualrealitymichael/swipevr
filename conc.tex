\section{Conclusion}

\subsection{Overcomes Latency and Proprioception with Swipe}
The lack of proproception in virtual reality is compounded by the latency.
For a user to be able to steer their finger to a key, the time needed to transmit information from the real to virtual and back to the ``real" through the  screen and into the eyes of the user is too long.

Therefore, we have to devices an alternate method of input that is not dependent on a feedback loop that requires the user to reach an absolute position on a keyboard.
Other input commonly used methods, such as gesture keyboard are subject to the same problem.

We device an alternate method that is only dependent on relative movement.
The user makes a swipe to one of six keys.
This reframes the domain of text input from a closed  feedback control problem to an open feedback control problem.
As such we are able to create a system for users that is able to efficiently enter text in virtual reality.


\subsection{Smooth Transition From Newbie to Novice to Expert }

A smooth pathway learnability is an important factor for an input device to gain widespread adoptability~\cite{}.
Some users will immediately reject an input device that requires too much upfront practice before it becomes useful.

SwipeVR offers a clear path from newbie to novice to expert. Newbie users are able to being using the system quickly by looking at the controller and slowly moving their thumb to a key.
As shown in the learnability chart, SwipeVR is instantly usable.
Novice users are able to periodically glance down or hold at the controller up within their field of view if they forget the position of a key.
Expert users are able to use memory to remember which way to move their thumb.
After some practice it becomes natural to memorize the location of the key by position.
At this point, the user is able to type, without looking down at the input device.
SwipeVR has a natural path to transition novice (low threshold) to expert (high ceiling) users~\cite{grover2013computational}.

\subsection{General Purpose Text Entry}
We hypothesize that virtual reality will be a general computing device much like the desktop or mobile phone.
While many of the apps in virtual reality will have their own specialized mechanism, we predict that at the operating system layer, there will be one general purpose text entry system that will be used across and only have to be learned how to use once.

This text entry system will need to be able to accommodate proper nouns, abbreviations, capital letters, colloquialism, passwords and other types of non-deterministic input.

SwipeVR is positioned well to accomodate these wide range of inputs and be the keyboard for a virtual reality operating system.

\subsection{Accessibility}
Virtual reality gives access access to something that they might never see in real life.
But for a disabled person, virtual reality might be the path to inclusion.
SwipeVR can be adapted for assistive technology products such as wands, joysticks, trackballs, touch screens.


\subsection{End}

concluding notes
\\
We present a fast text entry input device for virtual reality.
It is faster than the state of the art.
It can be used in public.
It dosnt tire your arms or vocal chords.

Using virtual reality for programming is a rich area for further discovery.
The large workspace that virtual reality provides is received well by users.

They keyboard is currently the weak point in many virtual reality applications.

Touch-typing on the keyboard is difficult as is switching context between the virtual reality controllers and the keyboard.

In the future, high-quality virtual reality headsets will be standalone devices, not tethered to desktop computers \cite{schaller1997moore}.
As the price of headsets decrease \cite{brown2016virtual}, virtual reality is positioned to become a democratizing technology.
Thus it is a worthwhile endeavour to determine how to program in virtual reality, both as a teaching tool for novices and as a more expressive tool for experts.

