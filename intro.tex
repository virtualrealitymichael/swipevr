\section{Introduction}

While the primary use of virtual reality is in games and videos today, it can be used in many application domains, from education, architectural reviews~\cite{guerreiro2014beyond}, healthcare, large group chats ~\cite{mcnerney1999system}, and data visualization~\cite{abbott2011empire}, exploration of space and other dangerous locations, scientific visualization, manufacturing, journalism, traveling, architecture, shopping.

Today's commercial virtual reality systems have relatively primitive input devices.
The most advanced controller available are the hand-tracked controllers provided by HTC Vive.
They enable users to interact with the objects in the virtual space, and it includes a trigger and a trackpad style click wheel.
The Oculus Rift currently uses a game controller, and the Samsung Gear has a couple of buttons on the headset.

This paper studies a very basic form of input in virtual reality environments: text.  There are many uses of text input in virtual reality, from typing in account names, passwords, search items, people you wish to contact, writing notes and chatting with others.    

The average typing performance on a regular keyboard was found to be about 60 words per minute\cite{Varcholik}, thanks to the dexterity of our fingers, the haptic feedback from the physical keyboards, and visual feedback on the display.    In the context of virtual reality, it is often not an option to use a physical keyboard.  We are usually not sitting down and it is hard to switch between the VR controllers and a physical keyboard, further complicated by the fact that we cannot see the keyboard in VR. It is desirable to be able to type with existing VR input devices, so we do not have to remove the head gear to type. 

Typing is challenging in virtual reality because of the latency in visual feedback and the lack of proprioception.  The industry standard is to enter text with gaze.  A virtual keyboard is presented in VR, and users type a letter by gazing at the desired key and tapping on the gear or controller.  Not only did the users find the process slow, but it can also causes nausea as they tap the head gear.  The user can type about 7 words per second.   

Entering text via speech is attractive, especially in the immersive virtual reality environment.   However, speech recognition is limited when typing passwords, names, and URLs~\cite{TODO}. 
The lack of user acceptance is also due to issues such as privacy, the perception of bothering others, the awkwardness of speaking to a machine~\cite{sawhney2000nomadic}.
Some users find voice as unreliable especially when the accent or ambient sound negatively affect accuracy.
Other alternatives proposed including using gesture, motion controllers, Leap Motion or tangibles, handheld controllers, or relying on a subset of keyboard and mouse commands~\cite{billinghurst1999collaborative}.
All these existing solutions, while potentially good for simple tasks, aren't adequate for tasks that requires a greater bandwidth of input~\cite{McGill:2015:DRO:2702123.2702382}.

For those situations where speech input is unacceptable, it is desirable to be able to enter text simply, quickly, and accurately on a touchpad, which is already built into the Vive controllers.   

\subsection{Informal Studies}
We started our research by trying some  possibilities.
\begin{enumerate}
\item
Gazing.  Users enter keys by gazing at the keys in a virtual keyboard. 

\item

Drumming.  One proposal is to display the keys as a virtual drum kit, and users enter keys by hitting the corresponding drums using the controllers as drumsticks. 

\item
Augmented reality.  We let users see their fingers and touchpad by streaming the view through a camera into virtual reality.  

\item
26-key tapping.  Users tap on the soft keyboard on the touchpad is shown as a virtual keyboard in virtual reality.   The users can see the keys being touched in the keyboard, as well as the words being type in a textbox.

\item
26-key swiping.  Users swipe on the soft keyboard on the touchpad instead.  
\end{enumerate}

The typing speed that we observed in these trials were in the range of 10-15 words per minute, far below our target performance.  Our initial trials reveal the deficiencies in each of these approaches.

\subsection{Contributions}
This paper proposes the Slide virtual keyboard, which uses three main design concepts: 
\begin{enumerate}
\item
Self-relative vectors.  To enter a character, the user slides from what is assumed to be the centroid of the keyboard to the key of interest.  

\item
Directional input, with error correction.  All keys are grouped in 6 tiles, and the user only has to slide in the direction of the tile containing the key.  A Bayesian word recommender determines the word entered. 

\item
Intention-based hybrid keyboard: the keyboard automatically switches between the 26-key and 6-tile keyboard based on the user's finger movements.  
\end{enumerate}

Our experiments show that …

In contrast, a recent study shows an average of 53 words per minute when typing on mobile devices\cite{landay}.

